\chapter{Modelling Approach}

In order to model the system sufficiently there are som decisions made on how
to model the system, and what to abstract away from. This chapter of the report
describes which decisions and the approach used in this model.

\section{Abstractions}
When it comes to abstracting things out of the model in this case it has mostly
been the underlying technologies used for e.g. encryption and hashing that has
been abstracted away from. Here instead unions of quote types is used to
construct flags that indicate what state the given data is supposed to be in.
This also makes it easier to validate that the model is doing what it should.
This is because it can be hard to check a model against encrypted or hashed
data. The abstractions used in this model is shown in
\cref{tbl:model:abstract}.

\begin{table}[H]
    \centering
    \begin{tabular}{|L{0.15\textwidth}|L{0.25\textwidth}|L{0.45\textwidth}|}\hline
        Class            & Abstraction    & Description                                                                                                                                                                                                                       \\ \hline
        Password         & Encryption     & {\ttfamily enqState} is used to keep track of whether or not the password is encrypted or not, while the password is always kept in plain text in the password member.                                                            \\ \hline
        PasswordDatabase & Encryption     & While the not only the passwords should be encrypted the whole database should so a third party cannot se how many passwords the user have. This is keep track of in the member {\ttfamily eqState}.                              \\\hline
        Device           & Identification & Right now the devices are just a having a token created on a running counter. To really identify a device MAC addresses, IP addresses, Cookies or other methods should probably be used to identify the device in the real world. \\\hline
        User             & Hashing        & The user password is stored in clear text even though it should be hashed to maintain security. Since this will not change anything in regards to how the model is working, this is abstracted away.                              \\\hline
    \end{tabular}
    \caption{The abstractions used in this model.}
    \label{tbl:model:abstract}
\end{table}

\section{Modelling Decisions}
Since this model revolves around checking that users can only access their data
there is a lot of things that is not included in this model either to keep the
scope clear like deciding not to model concurrency or because it is not
relevant to how the model works, like how devices should be authenticated
whether it is 2 factor authentication or something else.

This model does not take into account that the system probably will be a
distributed concurrent system and therefore, it might be necessary for the user
to have there own password database locally as well as online.

All user experiences in the system is not relevant for the model. This could be
things like auto filling passwords, how device authentication should happen, or
a user interface.